%LITERATURE REVIEW
\subsection{Zero Emission Neighbourhoods, ZEN, Definition, Goals and Ambitions}

%The Research Centre of Zero Emission Neighbourhoods in Smart Cities, ZEN, is a Norwegian initiative that started in 2017, following The Research Centre on Zero Emission Buildings, ZEB, that aims to \textit{"develop solutions for future buildings and neighbourhoods with no greenhouse gas emissions and thereby contribute to a low carbon society"} \cite{ZENwebsite}. Its goals touch on design and planning instruments, new busines models, creation of cost effective and resource and energy efficient buildings, technologies and solutions to operate energy flexible neighbourhoods, decision-support tool to optimize local energy systems and interactions, and creation and management of ZEN pilot projects in Norway \cite{ZENwebsite}.\note{Rephrase it better?}\\

The Research Centre of Zero Emission Neighbourhoods in Smart Cities define Zero Emission Neighborhoods as 


//How should the sustainable neighbourhoods of the future be designed, built, transformed, and managed to reduce their greenhouse gas (GHG) emissions towards zero?//


\subsection{ZEB, ZEN and Embodied Emissions}

- Findings of Zero emissions buildings ZEN LCA and MFA? studies/ or ZEB:


One common finding, however, is that buildings with low energy consumption present higher embodied emissions. Embodied emissions refer to the emissions upstream in the value chain due to the construction of the buildings, mostly due to the processing of materials used.  Moreover, in the best case scenarios, the energy generated by the ZEB is not enough to balance out these embodied emissions.  

- Findings of Zero emission neighbourhood studies: or ZEN:

	- Zero emissions concept, zero emission neighbourhood definition, what is desired from a ZEN. 

	Mobility, buildings, consumption, open spaces, facilities. 
	Ambitions. (LCA) 

	Before introducing the findings.. it is important to introduce the definition and ambitions of a ZEN. 


Even though ZEN studies are difficult to compare due to the particular functional units and boundaries set in each case, one interesting finding is that the use of transport in the neighbourhood has the higher emissions, followed by the production stage of the buildings or embodied emissions, (when considering passive houses). 

Meaning that the effect of transport in a neighbourhood is more harmful than the embodied emissions.

What are the measures to be prioritize? transport? buildings? infrastructure?


Energy measures in a neighbpurhood scale are important because they define the capacity of a neighbourhood to produce its own energy. It has been state that yearly energy consumption data of a neighbourhood/building does not allow to estimate reliable results because of the peak concept. In order to prove this a model has been developed by the ZEN research center that estimates the actual energy requirement, showing that in fact the not taking into account the coincidence factor leads to overestimation of the energy requirement.


*Lotteau, 2015.  Critical review of life cycle assessment (LCA) for the built environment at the neighborhood scale
	+ 21 existing cases of LCA at the N scale. 
	+ build knowledge to feed urban policy making or eco-design purpose. 
	+ built environment - refers to buidlings and transportation. summation of all human-made structures, infrastructure and trasportation systems. paper. buildings, open spaces (roads,green spaces), networks and mobility
	+ studies at the building scale - dominance of use phae, and increase of share of embodied energy for low-energy buildings. mostly process- base and LCEA, focus on energy issues. 
	+ city scale: 
	+ results.
	+ sensitivity of the results, operationals, mobility and embodied, range and sensitivity
	+ complexity of a neighbourhood, need for contextualization, 
	+ dynamic LCA, attributional vs. consequential. 
*Anderson 2015. 
	+ Structural materials, vs architectural materials. cement. industry
	+ Structural systems and structural materials. 
	+ induced impacts- interaction between individual buildings and urban context. 

- ZEN Report 2, findings and important details. 

\subsection{Materials in buildings}

- Refurbishment

\subsection{Circular Economy}


- Materials in buildings, passive buildings, difference in emissions from different type of buildings?. Refurbishment. 

	

- From zero emission buildings to zero emission neighbourhoods.
 Hotspots. Embodied emissions, transport. Difficulties and differences when modeling 
LCA results. What has been learned from LCA in ZEB and ZEN. 
		- Energy model hourly resolution of energy demand. Advantages. 

		To target this a dynamic model has been developed by the ZEN research center. This model 


- Circular economy? reduce embodied emissions? close the loop. h
	
	System thinking, prevent, reduce, reuse, recycle. 








- Zero emissions concept, zero emission neighbourhood definition, what is desired from a ZEN. 
Mobility, buildings, consumption, open spaces, facilities. 
Ambitions. (LCA) 

- From zero emission buildings to zero emission neighbourhoods.
 Hotspots. Embodied emissions, transport. Difficulties and differences when modeling 
LCA results. What has been learned from LCA in ZEB and ZEN. 
		- Energy model hourly resolution of energy demand. Advantages. 

- Materials in buildings, passive buildings, difference in emissions from different type of buildings?. Refurbishment. 

- Circular economy? reduce embodied emissions? 
 


\section{LCA and Dynamic MFA} % but also dynamic mfa?


Vilde Borgnes: 

-Balance boundaries, life stages
-Physical boundaries

More recent studies emphasise the importance of including the energy and associated GHG emissions embodied in the materials. Several authors concluded that especially when low-energy buildings are evaluated, the share of the emissions from the use phase decrease compared to other stages of the life cycle (Brown, Olsson et al. 2014) (Chastas, Theodosiou et al. 2016) (Kristjansdottir, Heeren et al. 2017).

The effect of the inclusion of infrastructure is also examined by other authors. In a study performed by Anderson, Wulfhorst et al. (2015), where a neighbourhood built following existing standards in Melbourne was examined, the results show that the contribution from the infrastructure constitutes about 17\% of the total embodied energy in the neighbourhood. As an example, they found that “power lines, supported by timber poles every 20 m are more energy intensive over 100 years than the combined concrete and steel in all footings of the buildings”.

An assumption of a fixed service life for all the components will lead to incorrect results because of the variations in the actual lifetimes. In some of the studies reviewed by Mastrucci, Marvuglia et al. (2017), the authors have chosen to let the service life differ among materials and building parts, as well as for refurbishment measures, to avoid this issue


When both operation and materials are considered, for all the three elements, the mobility is the major contribution to global warming in both scenarios, with a share of 54\% for scenario 1 and 50\% for scenario 2. The buildings contribute to 43\% and 47\% for scenario 1 and 2 respectively, and only 3\% of the global warming potential are caused by the open spaces (for both scenarios).
Moreover, when focusing on the life cycle stages, the materials (product stage + replacement) constitute the largest contribution to emissions with a share of 76\% for scenario 1 and 57\% for scenario 2.

Material efficiencies will increase, and emissions from production of materials will decrease together with the emission intensities in the future. 

When deciding what elements and life cycle phases to include in a ZEN definition, it is crucial to have a clear and substantiated understanding of what we want to achieve, and where in the life cycle perspective the major emissions sources is found. If the goal is to decrease the carbon footprint of the neighbourhood (or of the inhabitants in the neighbourhood) to a minimum level, the elements that lead to considerable amounts of emissions need to be addressed. 

Regarding the materials, it may be more accurate to use EPD data when calculating the emissions from the buildings, especially when considering a specific neighbourhood. The difference in the results when relying of this type of data, instead on using the Ecoinvent database should be assessed. A standardized way in getting the information of these emissions should be decided, to facilitate comparability between material- and provider choices. It is also crucial to decide the best practice when it comes to the replacements, both when considering the materials in the buildings and infrastructure, energy supply systems in the buildings (photovoltaic panels, heat pump etc.), and especially the replacements of the cars.

%\section{LCA in Neighbourhoods}



%\section{Materials in Buildings and Neighbourhoods}


%\section{Dynamic MFA in Buildings and Neighbourhoods}


%\section{Tools}

%\subsection{LCA}
%\subsection{Dynamic flow analysis}



%\section{}