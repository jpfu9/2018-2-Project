\chapter{Introduction}

\section{Background}
In the quest to an environmentally sustainable future, climate change is one of the challenges that greatly stands out. The increase of stressors released and concentrated in the atmosphere such as man-made %(anthropogenic)
greenhouse gas (GHG) emissions contribute to global warming and climate change. This in turn impacts not only human society but is transforming the patterns of the biosphere compromising the stability and life in Earth \cite{someone}.\\* %citation here. 
%/// I know these have more wings to it, social SDGs, developing countries, smart cities, the concept of sutainability... as a conclusion?/// it is not only Climate Change that is driving sustainability efforts.

Among the economic activities that contributes to climate change, the IPCC in its fifth assessment \cite{IPCC2015} reported that buildings are responsible for 18.4\% of total GHG emissions. Of those 12\% are indirect emissions mainly from the use of electricity, a share that can vary substantially due to the emission factors of the different types of energy production. When seen in terms of energy consumption, 32\% of global final energy is used by buildings.\\

Mitigation possibilities in terms of energy savings have been identified in the buildings sector, where solutions and technology are ready available. Passive house designs lower considerably the energy consumption of a building and if the house is set to generate energy, nearly or net zero emissions buildings (nZEB, ZEB) and positive energy buildings appear, in these cases the house is a source of green energy.\\
\note{*Try to unite this too paragraphs, they have similar ideas}\\

The potential of this sector stands out when compared to other sectors where mitigation strategies are more difficult to achieve. As a result, policies and efforts have been set to lower energy consumption and  emissions from this sector. The European Union has set into place the Energy Performance of Buildings Directive and the Energy Efficiency Directive and has established that by 2020 all new buildings should be constructed to be ZEB \cite{buildings directive}. \\

The building sector as part of a broader context when combined with mobility, open spaces and networks such as water, sewage, telecommunications, heating distribution and electricity distribution form the built environment \cite{Lotteau2015} \cite{Popovici2013}. This built environment can be looked at multiple scales, from neighbourhood to urban or city scale. Analyzing it is interesting because sustainability is addressed at a higher and more complex level where different systems and variables overlap. Exciting questions such as how to better design a neighbourhood so that its emissions are reduced towards zero? \cite{ZENJan} \cite{ZENDefinition}, what parts of the built environment contribute the most to the overall impact?, how to integrate capabilities so that impacts are reduced?, arise. \\

Multiple LCA assessments at a building level \cite{a}\cite{b}\cite{a}\ and at neighbourhood \cite{a}\cite{a} and urban level \cite{a}\cite{a}\ have informed and improved the knowledge about the different hotspots in the systems. From a building perspective is the operation of the building that has the major energy consumption and emissions. However, embodied energy and emissions from the extraction and processing of materials in the construction and refurbishment phases lead on when buildings fulfill passive house standards. At a neighbourhood level, it has been identified that when passive house standards are the rule, it is mobility what seems to have a higher impact followed by embodied energy and emissions\cite{Lottau2015} \cite{report2017ZEN}. However, conclusions are less straighforward due to the complexity and differences between purposes and definitions of built environment analysis\cite{Lottau2015} \cite{others}.\\

Answering all the arising questions and start developing solutions that resemble the sustainability goal in the built environment is a huge task and requires studying the different pieces separately and as a set so that a greater understanding emerge. \\

The Research Centre of Zero Emission Neighbourhoods in Smart Cities, ZEN, is a Norwegian initiative that started in 2017, following The Research Centre on Zero Emission Buildings, ZEB, that aims to \textit{"develop solutions for future buildings and neighbourhoods with no greenhouse gas emissions and thereby contribute to a low carbon society"} \cite{ZENwebsite}. Its goals touch on design and planning instruments, new busines models, creation of cost effective and resource and energy efficient buildings, technologies and solutions to operate energy flexible neighbourhoods, decision-support tool to optimize local energy systems and interactions, and creation and management of ZEN pilot projects in Norway \cite{ZENwebsite}.\note{Rephrase it better?}\\

Among the studies done by the research centre and as part of the analytical framework for design and planning of ZEN, WP1, ZEN report No. 2 \cite{ZENJan} released early in 2018 develops a dynamic model that was created and tested using Excel and Matlab. The model calculates the building stock model, the energy demand and GHG-emissions of a neighbourhood in time. This dynamic model uses a detailed initial stock characterized by archetypes (cohort, type of floor area and renovation state), renovation, demolition and future construction patterns; as well as energy carriers, delivered hourly energy intensity profiles per archetype and carbon intensities in time, as input. The output includes a detailed development of the stock over the years, its energy demand and GHG emissions. \\

One of the strengths of the model is that it calculates future energy demand in an hourly basis by means of a coincidental analysis, this method avoids overestimating the energy demand making the model more precise in terms of energy calculations. This precision is desired when design concepts that intend to couple energy generation from the neighbourhood with enery storage capacity from the mobility system are on the table.\note{This achieves more efficient use of energy, decarbonization and independent energy systems.} The model is also highly detailed in the way that treats buildings as individual objects, this means that the life of each building can be traced over the years. It is also flexible when defining the initial characteristics of the neighbourhood and parameters for the estimations in time. The GHG emission model used the energy model results and takes into account that carbon intensities can change in time. Finally, the model allows the evaluation of different scenarios.\\

\note{------AQUÍ VOY-------}\\
Considering the strengths of the model, its level of detail, the already existing input templates to describe a neighbourhood evolvement and the output that it provides, it has been considered to use the model and extend it to incorporate an evaluation of the material use by the different activities taking place in the neighbourhood throughout the years. Namely construction, renovation and demolition. Moreover, this material use can be further translate into carbon emissions.

Even though a great deal of the efforts have been concentrated at the energy level, reducing and leveling out the energy and emissions from the construction, maintenance and end of life stages of the buildings will not only reduce the environmental footprint of  buildings to its real minimum, but can further reduce other economic sectors environmental footprint. 

The IPCC in its fifth assessment, \cite{IPCCIndustry} reported that industry contributes 32\% to the global GHG emissions, where 11\% are indirect emissions. The majority of this emissions are attributed to the processing of materials into products. Close to half of this emissions are due to iron, steel and cement production, materials that are highly present in the built environment. Among the mitigation options proposed by the report material efficiency \note{in manufacturing and product design}, and reduced product and service demand \note{(product-service efficiency and service demand reduction)} stand out because of its applicability to the built environment. The report suggest reusing steel \cite{Cooper2012}\note{(from IPCC)}, reviewing the amount of materials used in commercial buildings and the frequency of replacement \cite{IPCCIndustry}, reducing the use of cement and reusing concrete in constructions, and extending the lifespan of buildings and infrastructure as mitigation measures. 

A tool that could assess in detail the material flow and associated carbon footprint of a neighbourhood throughout the years ....



The use of the model ...
\note{(could the model be use to include objects different than buildings?, energy devices? )}


Ultimately is the use of the whole model with real cases and attractive scenarios what will reveal the strengths and weaknesses of the model. Tha master theses could be the opportunity to adopt the model. 


\\ 
\\
%The newest advances in the field indicate that is possible to construct buildings with low carbon emissions. Design and material choices can reduce the energy consumption of a building to its minimun, moreover by incorporating heat pump, solar panels, energy from waste incineration and other energy technologies \note{refine expression} a building or set of buildings can generate enough energy to level out the energy consumption of its entire operation lifetime. This is indeed a impressive achievement in energy savings. A new challenge then arises which entails reducing and leveling out the energy and emissions from the construction, maintenance and end of life stages of the buildings. Resolving this challenge not only reduces the environmental footprint of  buildings to its real minimum, but can further reduce other economic sectors environmental footprint. 





%On the other hand, looking the it is also reported that transport contributes with 14\% to total GHG emissions, and industry 32\%, where 11\% are indirect emissions. Industry emissions are particularly interesting because they mostly entail the processing of materials into products and services. Mitigation options presented in the IPCC report include: energy efficiency, emissions efficieny, material efficiency in manufacturing and product design, product-service efficiency and service demand reduction. 

....  include material layer, benefits, purpose, how can it be connected to the model?


....There is a growing interest for the neighborhood scale in the field of urban sustainability assessment. It is a typical operational scale for urban development projects and integrates key levers for urban eco-design. Indeed, this change of scale is driven by the need to address district scale levers to design buildings and neighbor- hoods of higher environmental performance and to address key issues such as bioclimatic design, shared equipment (e.g. district heating), urban density or mobility issues.... Lotteau 2015

.... question that I want to address with this study.
			- either, embodied emissions of other systems as well? vs. energy benefits?
			- layer of material to a model that has been developed?
			- inform amount and types of materials?

- Importance of LCA and MFA, combined. 

\section{Problem Definition}

\section{LiteratureReview}
In this section important concepts and results found in the literature are reviewed as a framework in the development of this project report.
\input{literatureReview}





...uncertainty yet flexibility at the beginning of a project. 
..Think about the differences between LCA and dynamic modeling. 


.... ZEN GOALS
+ Develop neighbourhood design and planning instruments while integrating science-based knowledge on greenhouse gas emissions.
+ Create new business models, roles and services that address the lack of flexibility towards markets and catalyse the  development of innovations for a broader public use.
+ Create cost effective and resource and energy efficient buildings by developing low carbon technologies and construction systems based on lifecycle design strategies.
+ Develop technologies and solutions for the design and operation of energy flexible neighbourhoods.
+ Develop a decision-support tool for optimizing local energy systems and their interaction with the larger system.
+ Create and manage a series of neighbourhood-scale living labs, which will act as innovation hubs and a testing ground for the solutions developed in the ZEN Research Centre.

WP1- Analytical framework for design and planning of ZEN
WP2- Policy measures, innovation and business models. 
WP3- Responsive and energy efficient buildings. 
WP4- Energy flexible neighbourhoods. 
WP5- Local energy system optimization in a larger system
WP6- Pilot projects and living labs.
...





The ZEN








ZEB and PEB solutions 

 

-Neighbourhood dimension?
-



On the other hand, looking the it is also reported that transport contributes with 14\% to total GHG emissions, and industry 32\%, where 11\% are indirect emissions. Industry emissions are particularly interesting because they mostly entail the processing of materials into products and services. Mitigation options presented in the IPCC report include: energy efficiency, emissions efficieny, material efficiency in manufacturing and product design, product-service efficiency and service demand reduction. 




/ZEB definition: ZEB - "a building that has a very high energy performance, hereby the nearly zero or very low amount of energy required should be covered to a very significant extent by energy from renewable sources including energy from renewable sources produced on-site or nearby” (European Parliament and the Council 2010) /// This is from a report shared.. so dont copy./




Transport?


Consumption?


Circular economy?


